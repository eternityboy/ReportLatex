%%%%%%%%%%%%%%%%%%%%%%%%%%%%%%%%%%%%%%%%%
% ... vmv: modified template to be used in the Aalto University course
% ... MEC-E2012 - Computational Marine Hydrodynamics
% ...
% ... This is version 1.0, on September 10 2018 
% ... 
% ... Contact information: nikita.@vtt.fi
% ... Ville Viitanen, Tietotie 1A (VTT Ship Laboratory)
% ...
%----------------------------------------------------------------------------------------
%	PACKAGES AND OTHER DOCUMENT CONFIGURATIONS
%----------------------------------------------------------------------------------------

\documentclass[11pt]{article}
\usepackage[russian]{babel}
\usepackage[utf8]{inputenc}

\usepackage{amsmath}
\usepackage{graphicx}

\usepackage{fancyhdr}
\pagestyle{fancy}
\renewcommand{\headrulewidth}{0.4pt}  
\renewcommand{\footrulewidth}{0.4pt}

% пакет для использования ссылок в тексте
% описание https://en.wikibooks.org/wiki/LaTeX/Hyperlinks
\usepackage{hyperref}

% пакет для использования цветов в тексте
% описание https://en.wikibooks.org/wiki/LaTeX/Colors
\usepackage{xcolor}


\begin{document}
\begin{titlepage}

\newcommand{\HRule}{\rule{\linewidth}{0.2mm}} % Defines a new command for the horizontal lines, change thickness here

\center % Center everything on the page
 
%----------------------------------------------------------------------------------------
%	HEADING SECTIONS
%----------------------------------------------------------------------------------------
\textsc{Санкт-Петербургский государственный\\морской технический университет\\(СПбГМТУ)}\\ % Название университета

\HRule\\[0.5cm]
\textsc{\Large Кафедра Гидроаэромеханики\\и морской акустики}\\[0.5cm] % Название кафедры или факультета
\HRule\\[1cm]
%----------------------------------------------------------------------------------------
%	TITLE SECTION
%----------------------------------------------------------------------------------------


\huge{\bfseries{Шаблон Вашего курсового проекта или научно-исследовательской работы}}\\[1.5cm] % Название Вашего проекта
 
%----------------------------------------------------------------------------------------
%	AUTHOR SECTION
%----------------------------------------------------------------------------------------

\begin{minipage}{0.4\textwidth}
\begin{flushleft} \large
\emph{Автор:}\\
Иванович \textsc{И.И.} \\  % Ваше имя
\end{flushleft}
\end{minipage}
~
\begin{minipage}{0.4\textwidth}
\begin{flushright} \large
\emph{Группа:} \\
1380  \\ % Номер группы
\end{flushright}
\end{minipage}\\[2cm]
 
\begin{minipage}{0.4\textwidth}
\begin{flushleft} \large
\end{flushleft}
\end{minipage}
~ 
\begin{minipage}{0.4\textwidth}
\begin{flushright} \large
\emph{Проверил:} \\
к.т.н., ст. преп.\\
Тряскин \textsc{Н.В.} 
\end{flushright}
\end{minipage}\\[2cm] 
 

%----------------------------------------------------------------------------------------
%	DATE SECTION
%----------------------------------------------------------------------------------------

{\large \today}\\[0.5cm] % Дату можно изменить на конкретную, для этого следует изменить \today на например: 15 октября 2018 г.

% логотип СПбГМТУ
\includegraphics[height=50pt]{images/SMTU_Logo.png}
 
\vfill % Заполнить оставшееся место на страницы белым
\end{titlepage} % Конец титульной страницы


\section{Введение}\label{sec:Introduction}
	\begin{enumerate}
		\item Вводная часть вашего проекта.
		\item Что сделано и зачем.
		\item Обзор литературы по теме исследовательской работы
	\end{enumerate}

Пример написания таблицы в \LaTeX:
\begin{table}[htb]
\centering
\caption{Это пример таблицы.}
\label{Tab:TabExample} %метка с произвольным наименованием таблицы для ссылки из текста
\begin{tabular}{c|c|c} % три столбика с разделением |
\textbf{Наименование} & \textbf{Наименование} & \textbf{Наименование} \\
\hline
Это	&	Пример	&	таблицы  \\
\hline
Подробное	&	описание таблицы	& доступно   \\
\hline
A	&	B	& $\frac{1}{x}$ \\
\hline
Простая	&	таблица	& пусто \\
\hline
\end{tabular}
\end{table}

Пример использования таблицы показан в Таб. \ref{Tab:TabExample} и подробное описание доступно в поисковом запросе <<Google -> Latex Tabular >> или <<Google -> Latex Tabular*>> или по \textcolor{blue}{\href{https://en.wikibooks.org/wiki/LaTeX/Tables}{ссылке}}.	
	
Далее показан пример использования рисунка:
\begin{figure}[htb]
\centering
\includegraphics[totalheight=35.0mm]{images/SMTU_Logo.png}\\[1cm]  
\caption{Пример использования рисунка.}
\label{fig:FigExample}
\end{figure}	
	
Пример использования рисунка показан на Рис.~\ref{fig:FigExample}.	

Пример использования уравнения показан ниже:
%
\begin{equation}
\frac{\partial \rho \mathbf{U}}{\partial t } + \nabla \mathbf{U} \mathbf{U} = -\nabla p + \nabla \tau_{ij} + \rho \mathbf{g},
\label{eq:MomEquation}
\end{equation}

\begin{equation}
\nabla \cdot \mathbf{U}=0,
\label{eq:ContEquation}
\end{equation}
%
где $\rho$ это плотность, $\mathbf{U}$ вектор скорости, и ... Обязательно все элементы уравнения должны быть представлены, если не были представлены до этого! 

Уравнение Навье-Стокса (\ref{eq:MomEquation}) и уравнение неразрывности (\ref{eq:ContEquation}) представлено в векторно-тензорной форме.

\begin{equation}
\begin{cases}
	\begin{aligned}
		\frac{\partial u_{i}}{\partial t}+u_{j}\frac{\partial u_{i}}{\partial x_{j}} &=-\frac{1}{\rho}\frac{\partial p}{\partial x_{i}}+\frac{1}{\rho}\frac{\partial \tau_{ij}}{\partial x_{j}}\\
		\frac{\partial u_{j}}{\partial x_{j}} &=0
	\end{aligned}
\end{cases}
\label{eq:NSEquation}
\end{equation}
где $i=1,2,3$ и $j=1,2,3$

Уравнение Навье-Стокса и уравнение неразрывности представлено в скалярно-тензорной форме (\ref{eq:NSEquation}).

\section{Численные методы}\label{sec:NumMethod}

\begin{enumerate}
	\item Используемые уравнения движения жидкости.
	\item Используемые численные методы.
	\item Выбранная модель турублентности.
	\item Выбранные метод дискретизации каждой переменной.
	\item Сгенерированая численная сетка с её описание.
\end{enumerate}

\subsection{Использованные уравнения}\label{sec:FlowEqs}
Использованы уравнения Навье-Стокса и неразрывности представленные в разделе \ref{sec:Introduction}.

\subsection{Установка MikTex}\label{sec:MikTex}
Заходим на сайт \url{https://miktex.org/download} и скачиваем соответствующую версию пакета.

\subsection{Установка TexMaker}\label{sec:MikTex}
Заходим на сайт \url{http://www.xm1math.net/texmaker/} и скачиваем соответствующую версию пакета.

\section{Results\label{sec:Results}}
	\begin{enumerate}
		\item Была поставлена задача с начальными условиями:
		\item Решение задачи.
	\end{enumerate}
	
\section{Заключение}

	\begin{enumerate}
		\item Что было сделано.
		\item Выводы.
		\item Темы будущих исследований.
	\end{enumerate}
	
\section{Список Литературы} 

\end{document}
